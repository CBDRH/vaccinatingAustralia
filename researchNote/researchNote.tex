\documentclass{article}

\usepackage{arxiv}

\usepackage[utf8]{inputenc} % allow utf-8 input
\usepackage[T1]{fontenc}    % use 8-bit T1 fonts
\usepackage{lmodern}        % https://github.com/rstudio/rticles/issues/343
\usepackage{hyperref}       % hyperlinks
\usepackage{url}            % simple URL typesetting
\usepackage{booktabs}       % professional-quality tables
\usepackage{amsfonts}       % blackboard math symbols
\usepackage{nicefrac}       % compact symbols for 1/2, etc.
\usepackage{microtype}      % microtypography
\usepackage{lipsum}
\usepackage{graphicx}

\title{Vaccinating Australia: How long will it take?}

\author{
    Mark Hanly
   \\
    Centre for Big Data Research in Health \\
    UNSW Sydney \\
  Sydney 2052 \\
  \texttt{\href{mailto:m.hanly@unsw.edu.au}{\nolinkurl{m.hanly@unsw.edu.au}}} \\
   \And
    Tim Churches
   \\
    Ingham Institute for Applied Medical Research \\
    South Western Sydney Clinical School, Faculty of Medicine, UNSW Sydney \\
  Liverpool \\
  \texttt{\href{mailto:timothy.churches@unsw.edu.au}{\nolinkurl{timothy.churches@unsw.edu.au}}} \\
   \And
    Oisín Fitzgerald
   \\
    Centre for Big Data Research in Health \\
    UNSW Sydney \\
  Sydney 2052 \\
  \texttt{\href{mailto:o.fitzgerald@unsw.edu.au}{\nolinkurl{o.fitzgerald@unsw.edu.au}}} \\
   \And
    Louisa Jorm
   \\
    Centre for Big Data Research in Health \\
    UNSW Sydney \\
  Sydney 2052 \\
  \texttt{\href{mailto:l.jorm@unsw.edu.au}{\nolinkurl{l.jorm@unsw.edu.au}}} \\
  }


% Pandoc citation processing

\usepackage{booktabs}
\usepackage{longtable}
\usepackage{array}
\usepackage{multirow}
\usepackage{wrapfig}
\usepackage{float}
\usepackage{colortbl}
\usepackage{pdflscape}
\usepackage{tabu}
\usepackage{threeparttable}
\usepackage{threeparttablex}
\usepackage[normalem]{ulem}
\usepackage{makecell}
\usepackage{xcolor}


\begin{document}
\maketitle

\def\tightlist{}


\begin{abstract}
Add abstract here.
\end{abstract}

\keywords{
    COVID19
   \and
    vaccination
  }

\newpage

\hypertarget{introduction}{%
\section{Introduction}\label{introduction}}

The development and regulatory approval of multiple safe and efficacious
COVID-19 vaccines in less than a year is a truly remarkable achievement.
The logistical task of administering the vaccine rapidly and fairly to
billions of people around the world will be no less of a challenge.
National vaccination programs have commenced in many countries including
Israel, the United States and the United Kingdom. The Australian
government has entered into four agreements for the supply of COVID-19
vaccines (Table \ref{tab:agreements}), with a view to starting
distribution in mid-February. and an ambitious target to vaccinate the
adult community by the end of October. This gives around 37 weeks to
administer two doses each to some 20 million adult Australians.

\begin{table}[H]

\begin{threeparttable}
\caption{\label{tab:agreements}Australia’s vaccine agreements}
\centering
\begin{tabular}[t]{>{\raggedright\arraybackslash}p{3cm}>{\raggedright\arraybackslash}p{3cm}>{\centering\arraybackslash}p{1cm}>{\raggedright\arraybackslash}p{2cm}>{\raggedright\arraybackslash}p{5cm}}
\toprule
Name & Type & Doses (millions) & Schedule & Status\\
\midrule
\cellcolor{gray!6}{Pfizer/BioNTech} & \cellcolor{gray!6}{mRNA vaccine} & \cellcolor{gray!6}{10} & \cellcolor{gray!6}{2 doses 21 days apart} & \cellcolor{gray!6}{Provisionally approved by the TGA}\\
University of Oxford AstraZeneca & Viral vector vaccine & 54 & 2 doses 28 days apart & Phase 3 clinical trials\\
\cellcolor{gray!6}{Novavax} & \cellcolor{gray!6}{Protein vaccine} & \cellcolor{gray!6}{51} & \cellcolor{gray!6}{2 doses 21 days apart} & \cellcolor{gray!6}{Phase 3 clinical trials}\\
COVAX Facility & Assorted & 25 & Assorted & 9 candidate vaccines in various clinical trial stages\\
\bottomrule
\end{tabular}
\begin{tablenotes}
\small
\item [] Adapted from https://www.health.gov.au/initiatives-and-programs/covid-19-vaccines/about-covid-19-vaccines/australias-vaccine-agreements
\end{tablenotes}
\end{threeparttable}
\end{table}

The national rollout strategy divides the population into 16 groups,
organised into five phases (Table \ref{tab:phases}). Hospital hubs with
access cold chain storage facilities will administer the Pfizer/BioNTech
vaccine to the highest priority groups scheduled in Phase 1a, which
includes border workers, frontline healthcare staff, and aged care staff
and residents. Pending approval, the Astra Zeneca vaccine would be
administered to the bulk of the population thorugh a network of GPs and
pharmacies, The prime minister has quoted a target of 80,000
vaccinations per day,\textsuperscript{1} however a back of the envelope
calculation would suggest that at this rate it would take some 500 days
to vaccinate 20 million adult Australians twice, placing the finish line
somewhere around the middle of 2022. To reach the October target would
take closer to 160,000 daily vaccinations.

An added complication is that the Pfizer/BioNTech and AstraZeneca
vaccines both require two doses---a primer and a booster---which need to
be delivered within a specified time frame after the initial shot. This
complicates rollout as the daily resources of both vaccine units and
healthcare staff must be divided between those waiting for their first
jab and those returning for their booster.

\begin{table}[H]

\begin{threeparttable}
\caption{\label{tab:phases}Australia’s COVID-19 vaccine national roll-out strategy}
\centering
\begin{tabular}[t]{>{\raggedright\arraybackslash}p{1cm}>{\raggedright\arraybackslash}p{11cm}>{\raggedleft\arraybackslash}p{2cm}}
\toprule
Phase & Description & Size\\
\midrule
\textbf{\cellcolor{gray!6}{1a}} & \cellcolor{gray!6}{Quarantine \& border workers} & \cellcolor{gray!6}{70,000}\\
\textbf{1a} & Frontline health care workers & 100,000\\
\textbf{\cellcolor{gray!6}{1a}} & \cellcolor{gray!6}{Aged care and disability care staff} & \cellcolor{gray!6}{318,000}\\
\textbf{1a} & Aged care and disability care residents & 190,000\\
\textbf{\cellcolor{gray!6}{1b}} & \cellcolor{gray!6}{Elderly adults aged 80 years and over} & \cellcolor{gray!6}{1,045,000}\\
\textbf{1b} & Elderly adults aged 70-79 years & 1,858,000\\
\textbf{\cellcolor{gray!6}{1b}} & \cellcolor{gray!6}{Other health care workers} & \cellcolor{gray!6}{953,000}\\
\textbf{1b} & Aboriginal and Torres Strait Islander people aged 55 years and over & 87,000\\
\textbf{\cellcolor{gray!6}{1b}} & \cellcolor{gray!6}{Younger adults with an underlying medical condition} & \cellcolor{gray!6}{2,000,000}\\
\textbf{1b} & Critical and high risk workers & 196,000\\
\textbf{\cellcolor{gray!6}{2a}} & \cellcolor{gray!6}{Adults aged 60-69} & \cellcolor{gray!6}{2,650,000}\\
\textbf{2a} & Adults aged 50-59 & 3,080,000\\
\textbf{\cellcolor{gray!6}{2a}} & \cellcolor{gray!6}{Aboriginal and Torres Strait Islander people aged 18-54} & \cellcolor{gray!6}{387,000}\\
\textbf{2a} & Other critical and high risk workers & 453,000\\
\textbf{\cellcolor{gray!6}{2b}} & \cellcolor{gray!6}{Balance of adult population} & \cellcolor{gray!6}{6,643,000}\\
\textbf{3} & <18 if recommended & 5,670,000\\
\bottomrule
\end{tabular}
\begin{tablenotes}
\small
\item [] Adapted from https://www.health.gov.au/sites/default/files/documents/2021/01/australia-s-covid-19-vaccine-national-roll-out-strategy.pdf
\end{tablenotes}
\end{threeparttable}
\end{table}

Another unknown is the question of vaccine hesitancy, which refers to
the delay in acceptance or refusal to take a vaccine despite one being
available.\textsuperscript{2} Clearly, high levels of vaccine hesitancy
would have the potential to undermine efforts to vaccinate the
population. An online survey of over 3,000 Australian adults undertaken
in August 2020 asked respondents if they would take a safe and effective
vaccine for COVID-19, if one was developed. The population-weighted
responses were 5.5\% definitely not, 7.2 probably not, 28.7\% probably
and 58.5\% definitely.\textsuperscript{3}

The aim of this analysis is to estimate how long it will take to
distribute the two-dose COVID-19 vaccine schedule to the Australian
population. We consider a variety of scenarios based on the daily
vaccine capacity, the timeframe between the first and second dose and
the scale of vaccine hesitancy in the population. We conclude by
comparing daily vaccination rates from countries where vaccination
programs are already underway.

\hypertarget{methods}{%
\section{Methods}\label{methods}}

\hypertarget{population-and-priority-groups}{%
\subsection{Population and priority
groups}\label{population-and-priority-groups}}

Our analysis was based on the 16 priority groups and five phases
proposed by the Australian government (see Table \ref{tab:phases}). Th
implied population size is 25.7 million people, including 5.67 million
children and adolescents under the age of 18. We assumed that equal
priority was given to all groups within the same phase.

\hypertarget{vaccine-roll-out-projections}{%
\subsection{Vaccine roll out
projections}\label{vaccine-roll-out-projections}}

Roll out projections were based on four parameters:

\begin{enumerate}
\def\labelenumi{\arabic{enumi}.}
\tightlist
\item
  The daily vaccination capacity.
\item
  The minimum number of days between the first dose and the second dose.
\item
  The maximum number of days between the first dose and the second dose.
\item
  Vaccine hesitancy
\end{enumerate}

\hypertarget{projection-scenarios}{%
\subsection{Projection scenarios}\label{projection-scenarios}}

Projection scenarios were based on a \(2^k\) factorial design defined by
three factors with two levels each. The scenarios are summarised in
Table \ref{tab:scenarios} and the three factors and levels are detailed
below:

\begin{itemize}
\item
  Daily vaccination capacity (80,000 versus 170,000)
\item
  Timing between first and second dose (3-6 weeks versus 3-12 weeks)
\item
  Vaccine hesitancy (7\% versus 13\%).
\end{itemize}

\begin{table}[H]

\caption{\label{tab:scenarios}Projection scenarios}
\centering
\begin{tabu} to \linewidth {>{}l>{\raggedleft}X>{\raggedleft}X>{\raggedleft}X}
\toprule
Scenario & Capacity (units per day) & Gap between doses & Hesitancy\\
\midrule
\textbf{\cellcolor{gray!6}{1}} & \cellcolor{gray!6}{80,000} & \cellcolor{gray!6}{3 to 6 weeks} & \cellcolor{gray!6}{7\%}\\
\textbf{2} & 80,000 & 3 to 6 weeks & 13\%\\
\textbf{\cellcolor{gray!6}{3}} & \cellcolor{gray!6}{80,000} & \cellcolor{gray!6}{3 to 12 weeks} & \cellcolor{gray!6}{7\%}\\
\textbf{4} & 80,000 & 3 to 12 weeks & 13\%\\
\textbf{\cellcolor{gray!6}{5}} & \cellcolor{gray!6}{60,000} & \cellcolor{gray!6}{3 to 6 weeks} & \cellcolor{gray!6}{7\%}\\
\textbf{6} & 60,000 & 3 to 6 weeks & 13\%\\
\textbf{\cellcolor{gray!6}{7}} & \cellcolor{gray!6}{60,000} & \cellcolor{gray!6}{3 to 12 weeks} & \cellcolor{gray!6}{7\%}\\
\textbf{8} & 60,000 & 3 to 12 weeks & 13\%\\
\bottomrule
\end{tabu}
\end{table}

The projections assumed that those who were hesitant would never receive
the vaccine. The range of hesitancy rates were based on the survey data
reported by Edwards et al,\textsuperscript{3} and only applied to
general population groups. Border staff, healthcare and aged care
workers, aged care residents and adults with a medical condition were
assumed to have 0\% vaccine hesitancy.

\hypertarget{vaccine-allocation}{%
\subsection{Vaccine allocation}\label{vaccine-allocation}}

We allocated the daily available vaccination doses according to the
following algorithm:

\begin{enumerate}
\def\labelenumi{\arabic{enumi}.}
\item
  Calculate the number of second doses due, based on the specified
  permissible range for the second dose. As an example, if the second
  dose is is specified to be administered between 3 and 6 weeks, then
  the booster shots for people vaccinated on day 1 would be evenly
  distributed across the three weeks between day 22 and day 42).
\item
  Assign the remaining doses from the daily limit to those awaiting
  their first dose.
\item
  Identify the highest priority phase that haven't received all first
  doses.
\item
  Divide the available first doses between the subgroups in the highest
  priority phase, proportional to the number of unvaccinated individuals
  remaining in each subgroup.
\item
  Stop when all population members, minus those who are hesitant, have
  been vaccinated twice.
\end{enumerate}

\hypertarget{software-and-code}{%
\subsection{Software and code}\label{software-and-code}}

The analysis was performed using RStudio version 4.0.3. The source code
can be accessed at \url{https://github.com/CBDRH/vaccinatingAustralia}.

\hypertarget{results}{%
\section{Results}\label{results}}

Results from the eight scenarios are presented in Table
\ref{tab:scenarios}. Scenarios that assumed a vaccine hesitancy of 7\%
among the general population resulted in 48,337,780 vaccine doses
administered to 24,168,890 people, corresponding to a population
coverage of 94.0\%. Scenarios that assumed the vaccine higher hesitancy
rate of 13\% resulted in 45,713,020 vaccine doses administered to
22,856,510 people for a coverage of 88.9\%.

Under the most optimistic Scenario (Scenario 6), assuming a start date
of 15 February, Phase 1a would be fully vaccinated (i.e.~primer and
booster doses administered) as early as 30 March---just six weeks. The
entire adult population would be fully vaccinated by 11 October---in
line with government targets---and a further four weeks would see the
entire population include those under 18 vaccinated. Even under this
optimistic scenario, however, those in Phase 2a, including adults aged
60-69 years, would remain largely unvaccinated throughout the winter.
Under this scenario, we would reach 50\% population coverage in mid July
and 75\% population Around the start of October (Figure
\ref{fig:cumlResults}A).

\begin{table}[H]

\caption{\label{tab:projections}Summary of vaccine rollout projections for different scenarios}
\centering
\begin{tabu} to \linewidth {>{}l>{\raggedleft}X>{\raggedleft}X>{\raggedleft}X>{\raggedright}X>{\raggedleft}X>{\raggedleft}X>{\raggedleft}X>{\raggedright}X}
\toprule
Scenario & Number of vaccinations & Individuals vaccinated & Population coverage & Phase 1a complete & Phase 1b complete & Phase 2a complete & Phase 2b complete & Phase 3 complete\\
\midrule
\textbf{\cellcolor{gray!6}{1}} & \cellcolor{gray!6}{48,337,780} & \cellcolor{gray!6}{24,168,890} & \cellcolor{gray!6}{94.0} & \cellcolor{gray!6}{04/04/21} & \cellcolor{gray!6}{21/08/21} & \cellcolor{gray!6}{25/01/22} & \cellcolor{gray!6}{27/06/22} & \cellcolor{gray!6}{06/11/22}\\
\textbf{2} & 45,713,020 & 22,856,510 & 88.9 & 04/04/21 & 18/08/21 & 08/01/22 & 04/06/22 & 05/10/22\\
\textbf{\cellcolor{gray!6}{3}} & \cellcolor{gray!6}{48,337,780} & \cellcolor{gray!6}{24,168,890} & \cellcolor{gray!6}{94.0} & \cellcolor{gray!6}{16/05/21} & \cellcolor{gray!6}{25/09/21} & \cellcolor{gray!6}{24/02/22} & \cellcolor{gray!6}{30/07/22} & \cellcolor{gray!6}{09/12/22}\\
\textbf{4} & 45,713,020 & 22,856,510 & 88.9 & 16/05/21 & 21/09/21 & 09/02/22 & 05/07/22 & 06/11/22\\
\textbf{\cellcolor{gray!6}{5}} & \cellcolor{gray!6}{48,337,780} & \cellcolor{gray!6}{24,168,890} & \cellcolor{gray!6}{94.0} & \cellcolor{gray!6}{30/03/21} & \cellcolor{gray!6}{04/06/21} & \cellcolor{gray!6}{11/08/21} & \cellcolor{gray!6}{20/10/21} & \cellcolor{gray!6}{22/12/21}\\
\textbf{6} & 45,713,020 & 22,856,510 & 88.9 & 30/03/21 & 02/06/21 & 06/08/21 & 11/10/21 & 08/12/21\\
\textbf{\cellcolor{gray!6}{7}} & \cellcolor{gray!6}{48,337,780} & \cellcolor{gray!6}{24,168,890} & \cellcolor{gray!6}{94.0} & \cellcolor{gray!6}{11/05/21} & \cellcolor{gray!6}{19/06/21} & \cellcolor{gray!6}{12/09/21} & \cellcolor{gray!6}{20/11/21} & \cellcolor{gray!6}{21/01/22}\\
\textbf{8} & 45,713,020 & 22,856,510 & 88.9 & 11/05/21 & 18/06/21 & 06/09/21 & 07/11/21 & 07/01/22\\
\bottomrule
\end{tabu}
\end{table}

Under the less optimistic scenarios of 80,000 doses administered daily,
it would take until June or July 2022 to vaccinate the adult population
(Table \ref{tab:scenarios}). Under Scenario 1, we would reach 50\%
population coverage around January 2022 and 75\% population coverage in
July 2022 (Figure \ref{fig:cumlResults}A).

\begin{figure}

{\centering \includegraphics{researchNote_files/figure-latex/cumlResults-1} 

}

\caption{Cumulative vaccine doses over time}\label{fig:cumlResults}
\end{figure}

\hypertarget{discussion}{%
\section{Discussion}\label{discussion}}

To meet the target of vaccinating all adult Australians by the end of
October 2021 there will need to be in the order of 170,000 doses
delivered daily on average--a furious pace.

To put bounds on the feasibility of this target, a potentially
illuminating exercise is to compare to the vaccination rates achieved by
other countries who have already begun to roll out their vaccination
programs (Figure \ref{fig:dailyVac}. The stand out leader is Israel,
where between 7,000 and 20,000 vaccinations per million population have
been delivered daily through out January. Several factors have
contributed to this success, including a young, centralised population
and a strong public health infrastructure. Perhaps most important has
been strong logistical planning, including coordination of delivery,
cold-chain storage and staffing.\textsuperscript{4} Other countries have
been less successful in their rollout, including the United States
(4,000 per million pop), the United Kingdom (5,000 per million pop) and
the European Union (1,100 per million pop).

With a population of 25.7 million in Australia, the figure of 170,000
doses per day from our projections corresponds to around 6,600 daily
doses per million population. It will be possible to vaccinate the
Australian population in just nine months but we will need to do
considerably better than the rates currently being achieved in most
countries. We need to start now, get up to speed quickly and maintain
that pace. The invitation extended to community pharmacies to
participate in the vaccination rollout is a welcome development. There
are some 5,800 community pharmacies across Australia; if each could
vaccinate 15 community members a day that would represent half of the
daily target of 170,000 to cross the finish line for community
vaccination by the end of October.

\begin{figure}

{\centering \includegraphics{researchNote_files/figure-latex/dailyVac-1} 

}

\caption{Daily COVID-19 vaccines administered per one million population}\label{fig:dailyVac}
\end{figure}

\hypertarget{references}{%
\section*{References}\label{references}}
\addcontentsline{toc}{section}{References}

\hypertarget{refs}{}
\leavevmode\hypertarget{ref-pm2021}{}%
1. Press conference - australian parliament house. Published online
January 2021.
\url{https://www.pm.gov.au/media/press-conference-australian-parliament-house-12}

\leavevmode\hypertarget{ref-macdonald2015vaccine}{}%
2. MacDonald NE, others. Vaccine hesitancy: Definition, scope and
determinants. \emph{Vaccine}. 2015;33(34):4161-4164.

\leavevmode\hypertarget{ref-edwards2020covid}{}%
3. Edwards B, Biddle N, Gray M, Sollis K. COVID-19 vaccine hesitancy and
resistance: Correlates in a nationally representative longitudinal
survey of the australian population. \emph{medRxiv}. Published online
2020.

\leavevmode\hypertarget{ref-mckee2021can}{}%
4. McKee M, Rajan S. What can we learn from israel's rapid roll out of
covid 19 vaccination? \emph{Israel Journal of Health Policy Research}.
2021;10(1):1-4.

\bibliographystyle{unsrt}
\bibliography{references.bib}


\end{document}
